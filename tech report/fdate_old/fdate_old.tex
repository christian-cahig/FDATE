\documentclass[journal, a4paper]{IEEEtran}
%\IEEEoverridecommandlockouts
% The preceding line is only needed to identify funding in the first footnote. If that is unneeded, please comment it out.

\usepackage{cite}
\usepackage{amsmath,amssymb,amsfonts}
\usepackage{algorithm}
\usepackage{algorithmic}
\usepackage{bm}
\usepackage{graphicx}
\graphicspath{{../Figures/}}
\DeclareGraphicsExtensions{.pdf,.jpeg,.png}
\usepackage[caption=false,font=footnotesize]{subfig}
\usepackage{textcomp}
\usepackage[dvipsnames]{xcolor}
\usepackage{systeme}
%\usepackage{IEEEtrantools}

%% Custom Additions
\usepackage{array}
%\usepackage{caption,subcaption}
\usepackage{datetime}
\usepackage{mathtools}
\usepackage[hidelinks]{hyperref}
\usepackage{cleveref}
\usepackage{fontawesome5}
\usepackage{lipsum}
\usepackage{mleftright}
\usepackage{optidef}
\usepackage{physics}
\usepackage{xfrac}
%\usepackage[utf8]{inputenc}
%\usepackage[english]{babel}
%\usepackage{float}

% BibTeX
\def\BibTeX{{\rm B\kern-.05em{\sc i\kern-.025em b}\kern-.08em
		T\kern-.1667em\lower.7ex\hbox{E}\kern-.125emX}}

% Theorems
\newtheorem{theorem}{Theorem}[section]

% Floor and ceiling functions
\DeclarePairedDelimiter\ceil{\lceil}{\rceil}
\DeclarePairedDelimiter\floor{\lfloor}{\rfloor}

%% Front matter stuff
\title{Finite Difference Approaches to the\\Transmission Line Telegraph Equation}

%\author{
%	\IEEEauthorblockN{
%		Christian Y. Cahig\IEEEauthorrefmark{1}, Abdul Aziz G. Mabaning\IEEEauthorrefmark{2} \\
%	}
%	\IEEEauthorblockA{
%		\textit{Department of Electrical Engineering and Technology, College of Engineering and Technology}\\
%		\textit{Mindanao State University - Iligan Institute of Technology}\\
%		\IEEEauthorrefmark{1}chriscahig@gmail.com,
%		\IEEEauthorrefmark{2}mraaguevarra@gmail.com
%	}
%}

\author{
	Christian.~Y.~Cahig \textsuperscript{\faIcon[regular]{envelope}}, Michael~S.~Villame%
\thanks{
	C. Y. Cahig is with the
	Department of Electrical Engineering and Technology,
	Mindanao State University - Iligan Institute of Technology,
	Iligan City, Philippines.
	M. S. Villame is with the 
	<\textit{insert lab here}>,
	<\textit{insert uni here}>,
	<\textit{insert city here}>, Japan.
}%
\thanks{\faIcon[regular]{envelope} {\color{blue}christian.cahig@g.msuiit.edu.ph}}%
\thanks{\faIcon{github} Project page: {\color{blue}https://github.com/christian-cahig/telfindiff}}
}

%% Document
\begin{document}

\maketitle

\noindent
\textit{Updated as of {\today\ \currenttime}.}\\

\begin{abstract}
% TO DO: Update as often as necessary
\lipsum[4]
\end{abstract}

\begin{IEEEkeywords}
Finite difference method, telegraph equation, numerics, Python.
\end{IEEEkeywords}

\section{Introduction}
\label{sec: Introduction}

% TO DO: Quick background about numerical methods

% TO DO: What this paper is all about

% TO DO: Introduce succeeding sections
Section \ref{sec: Conclusion} concludes the work.

\section{The Telegraph Equation for Transmission Lines}
\label{sec: The Telegraph Equation for Transmission Lines}

% TO DO: Section introduction

\section{Finite Difference Methods for\\Solving the Telegraph Equation}
\label{sec: Finite Difference Methods for Solving the Telegraph Equation}

% TO DO: Section introduction

\section{Illustrative Examples}
\label{sec: Illustrative Examples}

% TO DO: Section introduction

\section{Conclusion}
\label{sec: Conclusion}

\lipsum[28]

%%%%%%%%%%%%%%%%%%%%%

% Acknolwedgement
%\vspace{40pt}
%\section*{Acknowledgment}

% References
%\vspace{40pt}
%\section*{References}
\bibliographystyle{IEEEtran}
%The argument is your BibTeX string definitions and bibliography database(s).
\bibliography{../References}

\end{document}
